% TikZ styles
\tikzstyle{header}=[draw, rectangle, minimum width=1.25em, minimum height=1.25em, anchor=south west]
\tikzstyle{block}=[draw, rectangle, minimum width=1.25em, minimum height=1.25em, fill=block_norm, anchor=south west]
\tikzstyle{header_bound}=[draw=none, minimum width=1.25em, minimum height=1.25em, anchor=south west, xshift=-0.4em, yshift=1em]
\tikzstyle{bound}=[draw=none, minimum width=1.25em, minimum height=1.25em, anchor=south west]
\tikzstyle{leveltag}=[rectangle, draw=none, rounded corners=1mm, fill=gray, text centered, anchor=north, text=white]

% TiKZ commands
\newcommand{\segmentationA}[2]{
\node[header] (a0) at (0,0) {#1};
  \xdef\lasti{0}
  \foreach [count=\i, remember=\i as \lasti] \x in {#2} {
    \node[block, minimum width=\x * 1.25 em] (a\i) [right=of a\lasti] {$\x$};
  }
}
\newcommand{\segmentationB}[1]{
\node[header] (b0) [below=of a0] {$s_2$};
  \xdef\lasti{0}
  \foreach [count=\i, remember=\i as \lasti] \x in {#1} {
    \node[block, minimum width=\x * 1.25 em] (b\i) [right=of b\lasti] {$\x$};
  }
}
\newcommand{\segmentationLabels}[1]{
\node[header_bound] (c0) [above=of a0] {};
  \xdef\lasti{0} %
  \foreach \xa/\xb/\xc/\xd/\xe [count=\i, remember=\i as \lasti] in {#1} {
    \node[bound] (c\i) [right=of c\lasti] {
    \begin{minipage}[b][3em]{0.5em}
    \centering
      \ifdefempty{\xe}{}{\xe \\}
      \ifdefempty{\xd}{}{\xd \\}
      \ifdefempty{\xc}{}{\xc \\}
      \ifdefempty{\xb}{}{\xb \\}
      \ifdefempty{\xa}{}{\xa}
    \end{minipage}
    };
  }
}
\newcommand{\segmentationLowerLabels}[1]{
\node[header_bound, yshift=-2em] (c0) [below=of a0] {};
  \xdef\lasti{0} %
  \foreach \xa/\xb/\xc/\xd/\xe [count=\i, remember=\i as \lasti] in {#1} {
    \node[bound] (c\i) [right=of c\lasti] {
    \begin{minipage}[t][3em]{0.5em}
    \centering
      \ifdefempty{\xe}{}{\xe \\}
      \ifdefempty{\xd}{}{\xd \\}
      \ifdefempty{\xc}{}{\xc \\}
      \ifdefempty{\xb}{}{\xb \\}
      \ifdefempty{\xa}{}{\xa}
    \end{minipage}
    };
  }
}
\newcommand{\boundariesOne}[1]{
  $
  \foreach \xa/\xb/\xc/\xd/\xe in {#1} {
    \{
      \ifdefempty{\xa}{{\color{white}0}}{\xa}
      \ifdefempty{\xb}{}{,\xb}
      \ifdefempty{\xc}{}{,\xc}
      \ifdefempty{\xd}{}{,\xd}
      \ifdefempty{\xe}{}{,\xe}
    \}
  }
  $
}

% listings settings
\lstset{ %
  language=,                % the language of the code
  basicstyle=\footnotesize,           % the size of the fonts that are used for the code
  numbers=left,                   % where to put the line-numbers
  numberstyle=\tiny\color{gray},  % the style that is used for the line-numbers
  stepnumber=1,                   % the step between two line-numbers. If it's 1, each line 
                                  % will be numbered
  numbersep=5pt,                  % how far the line-numbers are from the code
  backgroundcolor=\color{white},      % choose the background color. You must add \usepackage{color}
  showspaces=false,               % show spaces adding particular underscores
  showstringspaces=false,         % underline spaces within strings
  showtabs=false,                 % show tabs within strings adding particular underscores
  frame=single,                   % adds a frame around the code
  rulecolor=\color{black},        % if not set, the frame-color may be changed on line-breaks within not-black text (e.g. commens (green here))
  tabsize=2,                      % sets default tabsize to 2 spaces
  captionpos=b,                   % sets the caption-position to bottom
  breaklines=true,                % sets automatic line breaking
  breakatwhitespace=false,        % sets if automatic breaks should only happen at whitespace
  title=\lstname,                   % show the filename of files included with \lstinputlisting;
                                  % also try caption instead of title
  keywordstyle=\color{blue},          % keyword style
  commentstyle=\color{dkgreen},       % comment style
  stringstyle=\color{mauve},         % string literal style
  escapeinside={\%*}{*)},            % if you want to add a comment within your code
  morekeywords={*,...}               % if you want to add more keywords to the set
}